\documentclass[11pt,fleqn]{article}

\usepackage{color}
\usepackage{tikz}
\usetikzlibrary{positioning,shapes}

\definecolor{brown}{rgb}{0.7,0.2,0}
\definecolor{darkgreen}{rgb}{0,0.6,0.1}
\definecolor{darkgrey}{rgb}{0.4,0.4,0.4}
\definecolor{midgrey}{rgb}{0.9,0.9,0.9}
\definecolor{lightgrey}{rgb}{0.95,0.95,0.95}


\usepackage[spanish]{babel}
\usepackage{lmodern}
\usepackage[utf8]{inputenc}
\usepackage[T1]{fontenc}
\usepackage{listings}
\usepackage[colorlinks=true,urlcolor=blue]{hyperref}

\usepackage{framed}
\usepackage{verbatim}

\lstset{
   language=Haskell,
   gobble=2,
   frame=single,
   framerule=1pt,
   showstringspaces=false,
   basicstyle=\footnotesize\ttfamily,
   keywordstyle=\textbf,
   backgroundcolor=\color{lightgrey},
   literate={á}{{\'a}}1
            {é}{{\'e}}1
            {í}{{\'i}}1
            {ó}{{\'o}}1
            {ú}{{\'u}}1
            {ñ}{{\~n}}1
}

\newenvironment{answer}{%
  \def\FrameCommand{\fboxsep=\FrameSep \fcolorbox{black}{midgrey}}%
  \color{black}\MakeFramed {\FrameRestore}}%
 {\endMakeFramed}

\begin{document}

\title{CI4251 - Programación Funcional Avanzada \\
Solución de la tarea 6}

\author{Manuel Gómez\\
05-38235\\
\href{mailto:targen@gmail.com}{<targen@gmail.com>}}

\date{Julio 05, 2012}

\maketitle

\pagebreak

\section{Paralelismo}

El \textbf{colapso} de un número entero corresponde a la suma de
sus dígitos, proceso que puede seguirse recursivamente hasta
llegar a un dígito final. Por ejemplo, el colapso de 134957 es 29,
que a su vez colapsa en 11 que finalmente colapsa en 2.

Definiremos el colapso de una lista de enteros como el colapso
de la suma de los colapsos de los enteros que le componen.

\begin{itemize}
\item
  \emph{Baseline} \textbf{(1 punto)} -- escriba una función no paralela

\begin{lstlisting}
  collapse :: [Int] -> Int
\end{lstlisting}

  que calcule el colapso de una lista de enteros. Evalúe su
  desempeño trabajando sobre listas que tomen al menos dos
  minutos en ser procesadas hasta obtener el resultado
  (¡aproveche \verb=QuickCheck= con semillas manuales!).

\begin{answer}
  El cálculo del colapso de un entero se hizo explotando una fórmula
  cerrada sencilla usando el resto de la división entera.  Como se
  formuló la solución usando ejecutables separados para cada versión
  ---porque así resultaría más fácil medir el rendimiento en
  ejecución--- se abstrayó en un módulo aparte el comportamiento común.

\begin{lstlisting}
  module Collapse (collapse, main) where
    import System.Environment (getArgs)

    collapse :: Int -> Int
    collapse n = 1 + (n - 1) `mod` 9

    main :: ([Int] -> Int) -> IO ()
    main c = print.c.enumFromTo 0.read.head =<< getArgs
\end{lstlisting}

  \texttt{Collapse.main} no es un cómputo \texttt{IO} que pueda ser
  usado directamente como el cómputo principal de un ejecutable, pero
  abstrae el comportamiento de los tres ejecutables de esta solución.
  Se usa para calcular el colapso de la lista de todos los enteros desde
  el cero hasta un entero dado como argumento de línea de comandos.

  Aunque hubiera sido trivial ---\texttt{take <\$> (read . head <\$>
  getArgs) <*> (randoms <\$> newStdGen)} para generar muchos números
  aleatorios, o algo muy similar con \texttt{mkStdGen} para tomar
  también la semilla de un argumento de línea de comandos--- se decidió
  trabajar con listas de números secuenciales; lo que afecta el
  rendimiento es principalmente el tamaño de la lista, y la posibilidad
  de memorización de colapsos de números particulares no podría tener un
  efecto demasiado significativo considerando que su cálculo consiste de
  tres operaciones aritméticas.

  La implantación secuencial utiliza un \texttt{fold} estricto para
  calcular la suma de los colapsos de la lista.  No se usó simplemente
  \texttt{sum . map C.collapse} porque actualmente no se realiza fusión
  entre \texttt{sum} y \texttt{map}; hay algunos comentarios sobre este
  problema en \href{
    http://www.haskell.org/pipermail/haskell-cafe/2012-June/101971.html
  }{una discusión reciente sobre el tema en \textit{Haskell-cafe}}.

\begin{lstlisting}
  module Main (main) where
    import Data.Foldable (foldl')
    import qualified Collapse as C

    collapse :: [Int] -> Int
    collapse =
      C.collapse . foldl' (\ a n -> a + C.collapse n) 0

    main = C.main collapse
\end{lstlisting}
\end{answer}

\item
  \emph{Paralelismo implícito} \textbf{(4 puntos)} -- escriba la función

\begin{lstlisting}
  pcollapse :: [Int] -> Int
\end{lstlisting}

  que efectúe el mismo cálculo pero empleando paralelismo implícito,
  aprovechando las anotaciones paralelas de \texttt{Control.Parallel}.
  Si lo considera conveniente, puede definir estrategias \emph{ad hoc}
  en base a las primitivas de \texttt{Control.Parallel.Strategies}.

\begin{answer}
  Se siguió exactamente el esquema anterior, pero usando un \texttt{map}
  paralelo.  No se hizo nada más interesante (como limitar la creación
  de \textit{sparks}) porque no se tuvo acceso a un sistema SMP donde se
  pudiera hacer mediciones o pruebas.

\begin{lstlisting}
  module Main (main) where
    import Control.Parallel.Strategies (parMap, rpar)
    import qualified Collapse as C

    pcollapse :: [Int] -> Int
    pcollapse = C.collapse . sum . parMap rpar C.collapse

    main = C.main pcollapse
\end{lstlisting}
\end{answer}

\item
  \emph{Datos paralelos} \textbf{(4 puntos)} -- escriba la función

\begin{lstlisting}
  dcollapse :: [Int] -> Int
\end{lstlisting}

  que efectúe el mismo cálculo pero empleando paralelismo implícito,
  aprovechando las construcciones de \emph{Data Parallel Haskell}.
  Note que el tipo de \texttt{dcollapse} implica que debe aprovechar
  DPH a través de funciones auxiliares. Más aún, la solución
  \emph{correcta} usando DPH solamente requiere \textbf{una}
  función auxiliar.

\begin{answer}
  Se usó la implantación de \textit{Data Parallel Haskell} disponible en
  la versión 0.6.1.2 del paquete \texttt{dph-lifted-vseg} sobre GHC
  7.4.2.  Esta parte de la solución reside en dos módulos: uno con el
  cómputo \texttt{IO} principal del ejecutable, que no es vectorizable,
  y otro con la parte vectorizable.

  El módulo principal simplemente llama al módulo vectorizado con la
  lista generada usando el argumento de línea de comandos de la misma
  manera que en las otras dos partes de esta solución:

\begin{lstlisting}
  module Main (main) where
    import Data.Array.Parallel.PArray (fromList, toList)
    import System.Environment         (getArgs)

    import Vectorised (collapse)
    import qualified Collapse as C

    main = C.main $ collapse . fromList
\end{lstlisting}

  El módulo vectorizado exporta una función de interfaz que recibe el
  arreglo del módulo principal y lo transforma en un arreglo paralelo
  para poder trabajar con código vectorizado; el código vectorizado es
  quien efectúa el colapso de la suma de los colapsos del arreglo dado.

\begin{lstlisting}
  {-# OPTIONS -fvectorise #-}
  {-# LANGUAGE ParallelArrays, ParallelListComp #-}

  module Vectorised (collapse) where
    import Data.Array.Parallel (toPArrayP, fromPArrayP)
    import Data.Array.Parallel.PArray (PArray)
    import qualified Data.Array.Parallel.Prelude.Int as I

    {-# NOINLINE collapse #-}
    collapse :: PArray Int -> Int
    collapse xs = collapse' (fromPArrayP xs)

    collapse' :: [:Int:] -> Int
    collapse' ns=collapse1 (I.sumP [:collapse1 n | n<-ns:])

    collapse1 :: Int -> Int
    collapse1 n = 1 I.+ (n I.- 1) `I.mod` 9
\end{lstlisting}
\end{answer}
\end{itemize}

Para terminar \textbf{(1 punto)}, compare el desempeño de las funciones
\texttt{collapse}, \texttt{pcollapse} y \texttt{dcollapse} operando sobre
uno y dos núcleos. Procure ajustar \texttt{pcollapse} para obtener el máximo
desempeño posible sobre \textbf{dos} núcleos.

Es suficiente sustentar su comparación empleando las estadísticas emitidas
por el \emph{Runtime System}, sin embargo Ud. es libre de usar
\texttt{Criterion} si lo desea. En todo caso, los tres mejores
\emph{speedups} en mi máquina, recibirán tres (3), dos (2) y un (1) punto
adicional, respectivamente.

\begin{answer}
  No se pudo disponer de un sistema SMP donde pudieran hacerse medidas
  del comportamiento de esta solución en presencia de paralelismo.
\end{answer}
\end{document}
